\documentclass[landscape]{report}
\usepackage{bytefield}
\usepackage[landscape]{geometry}
\usepackage{listings}

\setlength\parindent{0pt}

\begin{document}
	
	\section*{File Format}
	
		\begin{bytefield}{64}
			\begin{rightwordgroup}{Header}
				\bitheader{0,7,15,23,31,39,47,55,63} \\
				\wordbox{1}{Magic Number Identifier} \\
				\bitbox{32}{Version Number} & \bitbox{32}{\textit{spare}} \\
				\bitbox{32}{Data Length} & \bitbox{32}{Text Length} \\
				\wordbox{1}{File Checksum}
			\end{rightwordgroup} \\
			\begin{rightwordgroup}{Data}
				\wordbox[lrt]{1}{} \\
				\skippedwords \\
				\wordbox[lrb]{1}{}
			\end{rightwordgroup} \\
			\begin{rightwordgroup}{Text}
				\wordbox[lrt]{1}{} \\
				\skippedwords \\
				\wordbox[lrb]{1}{}
			\end{rightwordgroup}
		\end{bytefield}
		
		\newpage
	
	\section*{Field descriptions}
	
		\subsection*{Magic Number Identifier}
	
		\begin{lstlisting}
0xF0 0x9F 0x8D 0x8C 0xF0 0x9F 0x8D 0x8C
		\end{lstlisting}
	
		A unique 64-bit value to mark the beginning of a \verb|.bb| file. Value is two copies of the UTF-8 encoding for the ``Banana'' emoji concatenated.
		
		\subsection*{Version Number}
		
		BNA version for this file. Not currently used.
		
		\subsection*{Data Length}
		
		Length of the Data segment in bytes.
		
		\subsection*{Text Length}
		
		Length of the Text segment in bytes.
	
		\newpage
	
	\section*{Data Segment}
	
		\subsection*{Entry Format}
		
		\begin{bytefield}{64}
			\bitbox{16}{identifier} &
				\bitbox{24}{\textit{spare}} &
				\bitbox{8}{type} &
				\bitbox{16}{data length} \\
			\wordbox[lrt]{1}{data} \\
			\skippedwords \\
			\wordbox[lrb]{1}{}
		\end{bytefield}

		\subsubsection*{Identifier}

		Identifier for this value. Used in instructions to find the value of this literal.

		\subsubsection*{Type}

		Data type of the value(s). % TODO determine types
		
		\subsubsection*{Length}
		
		Length of the data entry in \textbf{bytes}.
			
		\newpage
		
	\section*{Text Segment}
	
		\subsection*{Instruction Format}
		
			\vspace*{10pt}
			\begin{bytefield}{64}
				\bitheader{0,7,15,23,31,39,47,55,63} \\
				\bitbox{8}{op code} &
				\bitbox{16}{operand 1} &
				\bitbox{5}{\textit{spare}} &
				\bitbox{2}{\tiny op2 type} &
				\bitbox{32}{operand 2} \\
			\end{bytefield}
	
			\subsubsection*{op code}
			
			Identifier for the instruction to execute. Two special values, \verb|0x00| exits the program and \verb|0xFF| represents an invalid or unknown instruction (possibly will use for NOP later).
			
			\subsubsection*{operand 1}
			
			Identifier for the first operand of the instruction. Special value \verb|0x00| is \verb|null| and is for instructions with no first operand.
			
			\subsubsection*{op2 type}
			
			The type of operand in the \verb|operand 2| field. Uses the below table for the type values.
			
			\vspace{10pt}
			\begin{tabular}{l|c|l}
				\textbf{Code} & \textbf{Type} & \textbf{Value} \\ 
				\hline 
				\verb|0x00| & None & There is no second operand for this operation. \\ 
				\hline 
				\verb|0x01| & Variable & Stored as a variable, with \verb|operand 2| as the identifier. \\ 
				\hline 
				\verb|0x10| & Literal & Stored as a literal, with \verb|operand 2| as the identifier. \\ 
				\hline 
				\verb|0x11| & Small Literal & Value is \verb|operand 2| as a sign-extended integer. \\ 
			\end{tabular} 
			
			
			\subsubsection*{operand 2}
	
			Identifier or value for the second operand. See \verb|op2 type|
	
	
\end{document}